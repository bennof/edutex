%    EduTeX is a replacment and/or TeX lib for education
%    Copyright (C) 2010-2015  Benjamin Falkner
%
%    This program is free software: you can redistribute it and/or modify
%    it under the terms of the GNU General Public License as published by
%    the Free Software Foundation, either version 3 of the License, or
%    (at your option) any later version.
%
%    This program is distributed in the hope that it will be useful,
%    but WITHOUT ANY WARRANTY; without even the implied warranty of
%    MERCHANTABILITY or FITNESS FOR A PARTICULAR PURPOSE.  See the
%    GNU General Public License for more details.
%
%    You should have received a copy of the GNU General Public License
%    along with this program.  If not, see <http://www.gnu.org/licenses/>.
%


% plain TeX default file -- is replaced 

\catcode`\{=1 % left brace is begin-group character
\catcode`\}=2 % right brace is end-group character
\catcode`\$=3 % dollar sign is math shift
\catcode`\&=4 % ampersand is alignment tab
\catcode`\#=6 % hash mark is macro parameter character
\catcode`\^=7 \catcode`\^^K=7 % circumflex and uparrow are for superscripts
\catcode`\_=8 \catcode`\^^A=8 % underline and downarrow are for subscripts
\catcode`\^^I=10 % ascii tab is a blank space
\chardef\active=13 \catcode`\~=\active % tilde is active
\catcode`\^^L=\active \outer\def^^L{\par} % ascii form-feed is "\outer\par"


% Here is a list of the characters that have been specially catcoded:
\def\dospecials{\do\ \do\\\do\{\do\}\do\$\do\&%
  \do\#\do\^\do\^^K\do\_\do\^^A\do\%\do\~}

\catcode`@=11

\pdfoutput=1


\message{    EduTeX  Copyright (C) 2010-2015  Benno Falkner
    This program comes with ABSOLUTELY NO WARRANTY
    This is free software, and you are welcome to redistribute it
    under certain conditions}



\message{EduTex loading core element: codes,}
\input edutex.codes.tex
\message{registers,}
\input edutex.registers.tex
\message{parameters,}
\input edutex.parameter.tex
\message{fonts,}
\input edutex.fonts.tex
\message{macros,} 
\input edutex.macros.tex
\message{variables} %added
\input edutex.var.tex
\message{math definitions,}
\input edutex.math.tex
\message{output routines,} 
\input edutex.output.tex
\message{hyphenation,}
\input edutex.hyphenation.tex
\message{grades,} %added
\input edutex.grades.tex
\message{sheets,} %added
\input edutex.sheet.tex

\catcode`@=12
\endinput
