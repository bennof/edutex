%    EduTeX is a replacment and/or TeX lib for education
%    Copyright (C) 2010-2015  Benjamin Falkner
%
%    This program is free software: you can redistribute it and/or modify
%    it under the terms of the GNU General Public License as published by
%    the Free Software Foundation, either version 3 of the License, or
%    (at your option) any later version.
%
%    This program is distributed in the hope that it will be useful,
%    but WITHOUT ANY WARRANTY; without even the implied warranty of
%    MERCHANTABILITY or FITNESS FOR A PARTICULAR PURPOSE.  See the
%    GNU General Public License for more details.
%
%    You should have received a copy of the GNU General Public License
%    along with this program.  If not, see <http://www.gnu.org/licenses/>.

% based on plain TeX
\def\frenchspacing{\sfcode`\.\@m \sfcode`\?\@m \sfcode`\!\@m
  \sfcode`\:\@m \sfcode`\;\@m \sfcode`\,\@m}
\def\nonfrenchspacing{\sfcode`\.3000\sfcode`\?3000\sfcode`\!3000%
  \sfcode`\:2000\sfcode`\;1500\sfcode`\,1250 }

\def\normalbaselines{\lineskip\normallineskip
  \baselineskip\normalbaselineskip \lineskiplimit\normallineskiplimit}

\def\^^M{\ } % control <return> = control <space>
\def\^^I{\ } % same for <tab>

\def\lq{`} \def\rq{'}
\def\lbrack{[} \def\rbrack{]}

\let\endgraf=\par \let\endline=\cr

\def\space{ }
\def\empty{}
\def\null{\hbox{}}

\let\bgroup={ \let\egroup=}

% In \obeylines, we say `\let^^M=\par' instead of `\def^^M{\par}'
% since this allows, for example, `\let\par=\cr \obeylines \halign{...'
{\catcode`\^^M=\active % these lines must end with %
  \gdef\obeylines{\catcode`\^^M\active \let^^M\par}%
  \global\let^^M\par} % this is in case ^^M appears in a \write
\def\obeyspaces{\catcode`\ \active}
{\obeyspaces\global\let =\space}

\def\loop#1\repeat{\def\body{#1}\iterate}
\def\iterate{\body \let\next\iterate \else\let\next\relax\fi \next}
\let\repeat=\fi % this makes \loop...\if...\repeat skippable

\def\thinspace{\kern .16667em }
\def\negthinspace{\kern-.16667em }
\def\enspace{\kern.5em }

\def\enskip{\hskip.5em\relax}
\def\quad{\hskip1em\relax}
\def\qquad{\hskip2em\relax}

\def\smallskip{\vskip\smallskipamount}
\def\medskip{\vskip\medskipamount}
\def\bigskip{\vskip\bigskipamount}

\def\nointerlineskip{\prevdepth-1000\p@}
\def\offinterlineskip{\baselineskip-1000\p@
  \lineskip\z@ \lineskiplimit\maxdimen}

\def\topglue{\nointerlineskip\vglue-\topskip\vglue} % for top of page
\def\vglue{\afterassignment\vgl@\skip@=}
\def\vgl@{\par \dimen@\prevdepth \hrule height\z@
  \nobreak\vskip\skip@ \prevdepth\dimen@}
\def\hglue{\afterassignment\hgl@\skip@=}
\def\hgl@{\leavevmode \count@\spacefactor \vrule width\z@
  \nobreak\hskip\skip@ \spacefactor\count@}

\def~{\penalty\@M \ } % tie
\def\slash{/\penalty\exhyphenpenalty} % a `/' that acts like a `-'

\def\break{\penalty-\@M}
\def\nobreak{\penalty \@M}
\def\allowbreak{\penalty \z@}

\def\filbreak{\par\vfil\penalty-200\vfilneg}
\def\goodbreak{\par\penalty-500 }
\def\eject{\par\break}
\def\supereject{\par\penalty-\@MM}

\def\removelastskip{\ifdim\lastskip=\z@\else\vskip-\lastskip\fi}
\def\smallbreak{\par\ifdim\lastskip<\smallskipamount
  \removelastskip\penalty-50\smallskip\fi}
\def\medbreak{\par\ifdim\lastskip<\medskipamount
  \removelastskip\penalty-100\medskip\fi}
\def\bigbreak{\par\ifdim\lastskip<\bigskipamount
  \removelastskip\penalty-200\bigskip\fi}

\def\line{\hbox to\hsize}
\def\leftline#1{\line{#1\hss}}
\def\rightline#1{\line{\hss#1}}
\def\centerline#1{\line{\hss#1\hss}}

\def\rlap#1{\hbox to\z@{#1\hss}}
\def\llap#1{\hbox to\z@{\hss#1}}

\def\m@th{\mathsurround\z@}
\def\underbar#1{$\setbox\z@\hbox{#1}\dp\z@\z@
  \m@th \underline{\box\z@}$}

\newbox\strutbox
\setbox\strutbox=\hbox{\vrule height8.5pt depth3.5pt width\z@}
\def\strut{\relax\ifmmode\copy\strutbox\else\unhcopy\strutbox\fi}

\def\hidewidth{\hskip\hideskip} % for alignment entries that can stick out
\def\ialign{\everycr{}\tabskip\z@skip\halign} % initialized \halign
\newcount\mscount
\def\multispan#1{\omit \mscount#1\relax
  \loop\ifnum\mscount>\@ne \sp@n\repeat}
\def\sp@n{\span\omit\advance\mscount\m@ne}

\newif\ifus@ \newif\if@cr
\newbox\tabs \newbox\tabsyet \newbox\tabsdone

\def\cleartabs{\global\setbox\tabsyet\null \setbox\tabs\null}
\def\settabs{\setbox\tabs\null \futurelet\next\sett@b}
\let\+=\relax % in case this file is being read in twice
\def\sett@b{\ifx\next\+\def\nxt{\afterassignment\s@tt@b\let\nxt}%
  \else\let\nxt\s@tcols\fi \let\next\relax \nxt}
\def\s@tt@b{\let\nxt\relax \us@false\m@ketabbox}
\def\tabalign{\us@true\m@ketabbox} % non-\outer version of \+
\outer\def\+{\tabalign}
\def\s@tcols#1\columns{\count@#1\dimen@\hsize
  \loop\ifnum\count@>\z@ \@nother \repeat}
\def\@nother{\dimen@ii\dimen@ \divide\dimen@ii\count@
  \setbox\tabs\hbox{\hbox to\dimen@ii{}\unhbox\tabs}%
  \advance\dimen@-\dimen@ii \advance\count@\m@ne}

\def\m@ketabbox{\begingroup
  \global\setbox\tabsyet\copy\tabs
  \global\setbox\tabsdone\null
  \def\cr{\@crtrue\crcr\egroup\egroup
    \ifus@\unvbox\z@\lastbox\fi\endgroup
    \setbox\tabs\hbox{\unhbox\tabsyet\unhbox\tabsdone}}%
  \setbox\z@\vbox\bgroup\@crfalse
    \ialign\bgroup&\t@bbox##\t@bb@x\crcr}

\def\t@bbox{\setbox\z@\hbox\bgroup}
\def\t@bb@x{\if@cr\egroup % now \box\z@ holds the column
  \else\hss\egroup \global\setbox\tabsyet\hbox{\unhbox\tabsyet
      \global\setbox\@ne\lastbox}% now \box\@ne holds its size
    \ifvoid\@ne\global\setbox\@ne\hbox to\wd\z@{}%
    \else\setbox\z@\hbox to\wd\@ne{\unhbox\z@}\fi
    \global\setbox\tabsdone\hbox{\box\@ne\unhbox\tabsdone}\fi
  \box\z@}

\def\hang{\hangindent\parindent}
\def\textindent#1{\indent\llap{#1\enspace}\ignorespaces}
\def\item{\par\hang\textindent}
\def\itemitem{\par\indent \hangindent2\parindent \textindent}
\def\narrower{\advance\leftskip\parindent
  \advance\rightskip\parindent}

\outer\def\beginsection#1\par{\vskip\z@ plus.3\vsize\penalty-250
  \vskip\z@ plus-.3\vsize\bigskip\vskip\parskip
  \message{#1}\leftline{\bf#1}\nobreak\smallskip\noindent}
\outer\def\proclaim #1. #2\par{\medbreak
  \noindent{\bf#1.\enspace}{\sl#2\par}%
  \ifdim\lastskip<\medskipamount \removelastskip\penalty55\medskip\fi}

\def\raggedright{\rightskip\z@ plus2em \spaceskip.3333em \xspaceskip.5em\relax}
\def\ttraggedright{\tt\rightskip\z@ plus2em\relax} % for use with \tt only

\chardef\%=`\%
\chardef\&=`\&
\chardef\#=`\#
\chardef\$=`\$
\chardef\ss="19
\chardef\ae="1A
\chardef\oe="1B
\chardef\o="1C
\chardef\AE="1D
\chardef\OE="1E
\chardef\O="1F
\chardef\i="10 \chardef\j="11 % dotless letters
\def\aa{\accent23a}
\def\l{\char32l}
\def\L{\leavevmode\setbox0\hbox{L}\hbox to\wd0{\hss\char32L}}

\def\leavevmode{\unhbox\voidb@x} % begins a paragraph, if necessary
\def\_{\leavevmode \kern.06em \vbox{\hrule width.3em}}
\def\AA{\leavevmode\setbox0\hbox{!}\dimen@\ht0\advance\dimen@-1ex%
  \rlap{\raise.67\dimen@\hbox{\char'27}}A}

\def\mathhexbox#1#2#3{\leavevmode
  \hbox{$\m@th \mathchar"#1#2#3$}}
\def\dag{\mathhexbox279}
\def\ddag{\mathhexbox27A}
\def\S{\mathhexbox278}
\def\P{\mathhexbox27B}
\def\Orb{\mathhexbox20D}

\def\oalign#1{\leavevmode\vtop{\baselineskip\z@skip \lineskip.25ex%
  \ialign{##\crcr#1\crcr}}} \def\o@lign{\lineskiplimit\z@ \oalign}
\def\ooalign{\lineskiplimit-\maxdimen \oalign} % chars over each other
{\catcode`p=12 \catcode`t=12 \gdef\\#1pt{#1}} \let\getf@ctor=\\
\def\sh@ft#1{\dimen@#1\kern\expandafter\getf@ctor\the\fontdimen1\font
  \dimen@} % kern by #1 times the current slant
\def\d#1{{\o@lign{\relax#1\crcr\hidewidth\sh@ft{-1ex}.\hidewidth}}}
\def\b#1{{\o@lign{\relax#1\crcr\hidewidth\sh@ft{-3ex}%
    \vbox to.2ex{\hbox{\char22}\vss}\hidewidth}}}
\def\c#1{{\setbox\z@\hbox{#1}\ifdim\ht\z@=1ex\accent24 #1%
  \else\ooalign{\unhbox\z@\crcr\hidewidth\char24\hidewidth}\fi}}
\def\copyright{{\ooalign{\hfil\raise.07ex\hbox{c}\hfil\crcr\Orb}}}

\def\dots{\relax\ifmmode\ldots\else$\m@th\ldots\,$\fi}
\def\TeX{T\kern-.1667em\lower.5ex\hbox{E}\kern-.125emX}
\def\eduTeX{eduT\kern-.1667em\lower.5ex\hbox{E}\kern-.125emX}

\def\`#1{{\accent18 #1}}
\def\'#1{{\accent19 #1}}
\def\v#1{{\accent20 #1}} \let\^^_=\v
\def\u#1{{\accent21 #1}} \let\^^S=\u
\def\=#1{{\accent22 #1}}
\def\^#1{{\accent94 #1}} \let\^^D=\^
\def\.#1{{\accent95 #1}}
\def\H#1{{\accent"7D #1}}
\def\~#1{{\accent"7E #1}}
\def\"#1{{\accent"7F #1}}
\def\t#1{{\edef\next{\the\font}\the\textfont1\accent"7F\next#1}}

\def\hrulefill{\leaders\hrule\hfill}
\def\dotfill{\cleaders\hbox{$\m@th \mkern1.5mu.\mkern1.5mu$}\hfill}
\def\rightarrowfill{$\m@th\smash-\mkern-7mu%
  \cleaders\hbox{$\mkern-2mu\smash-\mkern-2mu$}\hfill
  \mkern-7mu\mathord\rightarrow$}
\def\leftarrowfill{$\m@th\mathord\leftarrow\mkern-7mu%
  \cleaders\hbox{$\mkern-2mu\smash-\mkern-2mu$}\hfill
  \mkern-7mu\smash-$}
\mathchardef\braceld="37A \mathchardef\bracerd="37B
\mathchardef\bracelu="37C \mathchardef\braceru="37D
\def\downbracefill{$\m@th \setbox\z@\hbox{$\braceld$}%
  \braceld\leaders\vrule height\ht\z@ depth\z@\hfill\braceru
  \bracelu\leaders\vrule height\ht\z@ depth\z@\hfill\bracerd$}
\def\upbracefill{$\m@th \setbox\z@\hbox{$\braceld$}%
  \bracelu\leaders\vrule height\ht\z@ depth\z@\hfill\bracerd
  \braceld\leaders\vrule height\ht\z@ depth\z@\hfill\braceru$}

\outer\def\bye{\par\vfill\supereject\end}
%\outer\def\end{\vfill\eject\csname bye\expandafter\endcsname}

%Link
\def\weblink(#1)#2{\hbox{\pdfstartlink user{/Subtype /Link /A << /Type /Action /S /URI/URI (#1) >>} #2\pdfendlink}}
\def\link(#1)#2{\hbox{\pdfstartlink goto page #1 {/Fit}  #2\pdfendlink}}

%Colors
\chardef\@ColorStack=0%\pdfcolorstackinit page direct{0 g 0 G}
%  define colors [rgb]
\def\defineColor(#1,#2,#3)#4{\expandafter\xdef\csname @ColorName@#4\endcsname##1{\pdfcolorstack \@ColorStack push {#1 #2 #3 rg} ##1 \pdfcolorstack \@ColorStack pop}}
\def\useColor#1{\csname @ColorName@#1\endcsname}

% Include Images
\def\includeImage[#1]#2{
\pdfximage width #1 {#2} \pdfrefximage\pdflastximage}


\def\scaleBox[#1]#2{\setbox0=\hbox{#2}
\pdfsave\pdfsetmatrix{#1 0 0 #1}\rlap{\smash{\copy0}}\pdfrestore
{\setbox2=\hbox{}\wd2=#1\wd0 \ht2=#1\ht0 \dp2=#1\dp0 \box2}
}


\def\PDF#1{\special{pdf:#1}}
\def\PDFD#1{\special{pdf:direct:#1}}
\def\PDF@Gpush{\PDF{q}}
\def\PDF@Gpop{\PDF{Q}}

% Latex
\def\makeatletter{\catcode`@=11}
\def\makeatother{\catcode`@=12}


% Additional macros
% new page
\def\newpage{\vfil\eject} 
% new line
\def\newline{\hfil\break}

% new variable
\def\newvar#1{\expandafter\xdef\csname #1\endcsname##1{\xdef\csname the#1\endcsname{##1}}\csname #1\endcsname{#1}} 


% convert to interger to letter [a-z]
\def\alphanumeral#1{\ifcase#1 ?\or a\or b\or c\or d\or e\or f\or g\or h\or i\or j\or k\or l\or m\or n\or o\or p\or q\or r\or s\or t\or u\or v\or w\or x\or y\or z\else xx\fi}
\def\monthName#1{\ifcase#1\or January\or February\or March\or April\or May\or June\or July\or August\or September\or October\or November\or December\fi}

%if star is added
\def\@ifstar#1#2{\let\@wstar#1 \let\@nstar#2 \futurelet\@next@char\@ifstarh}
\def\@ifstarh{\ifx\@next@char *\expandafter\@wstar\else\expandafter\@nstar\fi}


%if optional argument is given
\def\@optarg#1#2{\let\@woptarg#1 \let\@noptarg#2 \futurelet\@next@char\@optargh}
\def\@optargh{\ifx\@next@char [ \expandafter\@woptarg \else\expandafter\@noptarg\fi}

%remove pt from dimension
\edef\@rmpt{\def\noexpand\@rmpt##1.##2\string p\string t{##1.##2}} \@rmpt
\def\@strip@pt{\expandafter\@rmpt\the}

%remove pt from dimension and cut decimal
\edef\@intrmpt{\def\noexpand\@intrmpt##1.##2\string p\string t{##1}} \@intrmpt
\def\@dimtoint{\expandafter\@intrmpt\the}

\def\draft{\overfullrule=5pt}


